\documentclass[a4paper, 12pt]{article}
\renewcommand{\baselinestretch}{1.22}

% Packages
\usepackage[T1]{fontenc}
\usepackage[utf8]{inputenc}
\usepackage[ngerman]{babel}
\usepackage[ngerman=ngerman-x-latest]{hyphsubst}

\usepackage[a4paper, inner=2.5cm, outer=2.5cm, top=2.5cm, bottom=2.5cm, bindingoffset=0cm]{geometry}
\usepackage{csquotes}
\usepackage{graphicx}

\usepackage[citestyle=verbose-ibid,giveninits=true,doi=false,eprint=false,isbn=false,backend=biber]{biblatex}
\usepackage[bottom]{footmisc}
\addbibresource{bib/main.bib}
\DeclareNameAlias{default}{family-given}
\DeclareDelimFormat[bib]{nametitledelim}{\addcolon\space} % Doppelpunkt nach Autor

\graphicspath{{figures/}}

\begin{document}

\title{\vspace{-2.0cm}Sind meine Entscheidungen wirklich frei?} %oder maßgeblich beeinflusst?}
\author{Marvin Borner, Philosophie TG13}
\date{\today}

%{\parindent 0cm
%	\subsection*{Selbstständigkeitserklärung}
%	Ich erkläre hiermit, dass ich die vorliegende Arbeit selbstständig verfasst
%	und nur unter Verwendung der angegebenen Quellen und Hilfsmittel angefertigt habe.\\
%	\vspace{1\baselineskip}
%
%	Langenau, den \today \hspace{0.1\linewidth} \includegraphics[width=2cm]{unterschrift}
%}

\maketitle

\section{Einleitung}
Die Frage, ob meine Entscheidungen \textit{wirklich} frei sind, scheint im ersten Moment trivial beantwortbar zu sein, da die meisten Menschen vermutlich zustimmen würden, dass unsere Entscheidungen aus einem freien Willen heraus entstehen und somit eine volle Kontrolle aller Gedanken und Entscheidungen gegeben ist. Im Alltag wird dies oft sichtbar, wenn Menschen aufgrund ihrer Entscheidungen kritisiert werden und demnach davon ausgegangen wird, dass jene einen freien Willen besitzen.

Dennoch beschäftigen sich Philosophen vieler Generationen mit der Frage nach der Existenz der Willensfreiheit und kommen häufig nicht zu einer eindeutigen Antwort\footcite[S. 170]{walter}. Gerade in der modernen Zeit, in welcher sich Wissenschaften wie die Neurobiologie auf einer scheinbar eindeutig beweisbaren Ebene mit den Prozessen im Gehirn befassen, werden derartige Grundsatzdiskussionen dieser zwei Parteien, nämlich der Naturwissenschaft und der Geisteswissenschaft, erneut aufgegriffen. Auch wenn eine allumfassende Antwort auf die Frage offenbar nicht sehr einfach zu finden ist, werde ich mich im folgenden Essay mit ihr beschäftigen und einer Antwort annähern.

Damit eine Antwort gefunden werden kann, müssen grundlegende Begriffe zuerst genau definiert werden.

\section{Definitionen}
\subsection{Entscheidungen}
Eine \textit{Entscheidung} beschreibt prinzipiell eine Aufwertung verschiedener Argumente für mehrere Möglichkeiten und einen darauf folgenden Entschluss für eine dieser Möglichkeiten.

Eine \textit{freie} Entscheidung ist infolgedessen eine Entscheidung, welche aus einem freien Willen heraus entsteht. Dementsprechend hat ein Mensch, der einen freien Willen und somit auch eine Entscheidungsfreiheit besitzt, die Kontrolle über den Fortlauf einer Kausalkette und kann damit aktiv die Zukunft beeinflussen. Dabei stellt sich allerdings die Frage, was der freie Wille beziehungsweise die Freiheit überhaupt ist.

\subsection{Freiheit}
Bei der Aussage, dass \textit{Freiheit} ein nicht einfach zu beschreibender Begriff ist, würden wohl die meisten Menschen zustimmen. Denn nur selten wird ein solcher Begriff in so vielen Bereichen und Kulturen unterschiedlich definiert und verwendet wie dieser. Während Freiheit im politischen Sinne eine Unabhängigkeit von allen Unterdrückungen und Unterwerfungen wie Sklaverei und psychologischer Manipulation, oder auch eine Unabhängigkeit von politischen Institutionen sein kann, bezeichnet die Handlungsfreiheit die Fähigkeit, die Handlungen auszuführen, die dem freien Willen entsprechen, und jene Handlungen zu unterlassen, die diesem widersprechen.

Die Willensfreiheit ist hingegen schwieriger zu definieren: Diese beschreibt grundsätzlich die Fähigkeit, das entscheiden zu können was man will. Das beeinhaltet auch, dass die Entscheidungen nicht vorhersehbar sind und es dadurch unter den gleichen Bedingungen immer eine Möglichkeit gibt, sich anders zu entscheiden. Zudem kann ein freier Wille nur existieren, wenn die Person selbst die ausschließliche Kontrolle hat und damit der Urheber dieser Entscheidung ist und nicht von äußeren Faktoren dazu gezwungen oder gedrängt wird. Außerdem gilt, dass die Entscheidung nicht willkürlich geschehen darf, sondern aus \textit{bewussten} Absichten heraus entstanden ist, da diese sonst ein reines Zufallsereignis wäre. Nur wenn diese Bedingungen in jeder Hinsicht zutreffen, kann von einem vollkommenen freien Willen und somit auch von einer Entscheidungsfreiheit gesprochen werden.

\subsection{Determinismus}
Der \textit{Determinismus} beschreibt im Bezug auf die Willensfreiheit die Theorie, dass jedes Ereignis vorbestimmt ist und nur eine logische beziehungsweise aus den Naturgesetzen folgende Kausalität eines vorherigen Ereignisses ist. Diese Denkweise impliziert, dass aus jedem Zeitpunkt die Vergangenheit und die darauf folgende Zukunft vorhergesagt beziehungsweise zurückverfolgt werden kann. Der Determinismus bildet also intuitiv einen logischen Widerspruch zwischen dem Konzept der freien Entscheidung und Willensfreiheit, welche eine Veränderung dieser hier festen Kausalkette verlangen würde.

Im Gegensatz hierzu steht der \textit{Indeterminismus}.

% Bin wohl Inkompatibilist lol
\subsection{Zwischenstand}
Aufgrund dieser Definitionsansätze gilt demnach folgendes: Unbedingte Willensfreiheit und somit auch freie Entscheidungen kann es nur im Indeterminismus geben, da der Determinismus nicht die Bedingungen der Definition des freien Willens erfüllt.

\section{Beeinflussung}
Die Beeinflussung meiner Entscheidungsbildung spielt eine große Rolle, wenn es darum geht, diese als frei bezeichnen zu können. Denn Außenfaktoren wie das generelle Umfeld, die Vorlieben der Mitmenschen, Erziehung, zeitliche Präferenzen der Gesellschaft, kulturelle Eigenschaften und Aussagen der Medien und Gesetze, sowie empirische Faktoren\footnote{Als empirische Faktoren bezeichne ich diejenigen Variablen des Gehirns, die aus meinen Erfahrungen heraus entstanden sind, beziehungsweise von diesen verändert wurden.} wie Vorurteile und Ängste jeglicher Art können meine unterbewussten Bevorzugungen derart beeinflussen, dass meine Entscheidungen nahezu ausschließlich ein Produkt dieser Faktoren zu sein scheinen.

\section{Ich}
Bei der Frage, ob \textit{ich} nun einen freien Willen besitze oder nicht, kommt für mich die Frage auf, was dieses \textit{Ich} denn ist und was genau dazugehört. Wie soll man diese Frage schließlich beantworten können, wenn man nicht einmal genau weiß, was das Subjekt ist?

Im Alltag werden Milliarden an Entscheidungen innerhalb kürzester Zeit vom Gehirn durchgeführt, von denen das Bewusstsein gar nichts bemerkt. Wenn ich beispielsweise gewohnte Entscheidungen, wie die Wahl des Mittagessens oder die Wahl der Kleidung für den folgenden Tag, treffe, geschieht dies häufig unterbewusst und lässt mich nur glauben, dass diese Entscheidung von mir als Ganzes getroffen wurde, obwohl diese nur von automatisierten und neurophysischen Prozessen vollzogen wurde.

Ob diese kleinen unterbewussten Prozesse nun zu dem Ich dazugehören oder nicht, oder ob es sogar noch eine Zwischenebene -- einen Geist? -- gibt, welchen man als \textit{Ich} bezeichnen könnte, gehört zu einer der großen Debatten in der Anthropologie und Religion und werde auch ich nicht eindeutig beantworten können. Für mich ist allerdings klar, dass der Mensch nicht einzig und allein ein neurologisch verknüpfter Zellhaufen ist, sondern vielmehr ein von äußeren Faktoren geprägter Mechanismus, der die Fähigkeit besitzt, diese Einflüsse sowohl unterbewusst, als auch bewusst zu bewerten, zu vergleichen und gegebenenfalls zu akzeptieren, um die eigene Denkweise und Handlung an diese anzupassen.

Als Beispiel dazu dient die Abgabe dieses Essays. Während ich zwar die Möglichkeit hätte, ihn nicht abzugeben, weiß ich aufgrund von Erfahrungen dennoch, dass dies negative Auswirkungen auf meine Schulnote hätte. Somit werde ich quasi von diesem Wissen, meiner Erziehung und meinem Willen, eine gute Abschlussnote zu erreichen, welcher vermutlich auch nur das Ergebnis meiner Erziehung oder anderer empirischer Faktoren ist, dazu gedrängt, eine teils bewusste und teils unbewusste Evaluierung dieser Rahmenbedingungen durchzuführen und diesen Essay pünktlich abzugeben. Folglich ist jene Entscheidung nicht aus einem freien Willen heraus entstanden, sondern aus vielen Faktoren, welche in meinem Gehirn zu einem Entschluss geformt wurden.

\section{Illusion}
Wenn man nun die Prozesse des Gehirns und deren Beeinflussung aufgrund der genannten Faktoren näher betrachtet, sieht man, dass der freie Wille anscheinend nur eine Illusion ist. Wenn man noch einmal die Definition der Willensfreiheit in Erinnerung ruft, kann man nämlich erkennen, dass sowohl meine Entscheidungen -- zumindest zu einem gewissen Grad mithilfe von genauen Informationen, wie meine Neuronen bis heute beeinflusst wurden -- \textit{vorhersehbar} sind, als auch der Urheber meiner Entscheidung nicht allein ich bin, sondern beispielsweise auch die Personen, die mir bestimmte Fähigkeiten und Vorlieben anerzogen haben.

Der Mensch, beziehungsweise dessen Gehirn, kann demnach zwar entscheiden, sprich empirische Argumente und Faktoren miteinander vergleichen, aber nicht aus einem vollkommen freien Willen heraus. Menschliche Entscheidungen können folglich auch nicht frei genannt werden.

Der Psychologe Wolfgang Prinz sagt passend dazu: \enquote{Wir tun nicht, was wir wollen, sondern wir wollen, was wir tun}\footcite{prinz}. Diese Aussage entstand aus dem Ergebnis des bahnbrechenden Versuchs von dem Neuropsychologen Benjamin Libet, durch den herausgefunden wurde, dass der unterbewusste Bereich des Gehirns, der Körperteile steuert, bereits bis zu 400 Millisekunden vor der tatsächlichen Bewegung aktiviert wird.\footcite{libet} Je nachdem, ob man diese unterbewussten Gehirnaktivitäten zum bewussten Ich dazu zählt oder nicht, ist auch dieser Versuch ein Argument dafür, dass meine Entscheidungen determiniert beziehungsweise unabhängig vom bewussten Ich sind und dieses dementsprechend keinen freien Wille besitzt.

Der freie Wille ist folglich vielmehr eine Illusion, die sich über jahrelange soziale Erziehung aus Aussagen wie \enquote{Warum hast du das gemacht?} und \enquote{Du hättest dich auch anders entscheiden können!}, bei denen davon ausgegangen wird, dass eine Willensfreiheit als solches existiert, in unseren Köpfen verfestigt hat.

\section{Folgen}
Die Folgen einer allgegenwärtig anerkannten Inexistenz des freien Willens wäre allerdings katastrophal für unsere Gesetze und unser Weltverständnis. Da wir Menschen schon immer mit der Illusion des freien Willens aufgewachsen sind, sind auch unsere Gesetze derartig aufgebaut: Wenn man eine Straftat begeht, dann wird man unter anderem dafür bestraft, weil davon ausgegangen wird, dass man auch anders hätte handeln können und aktiv die Wahl getroffen hat, diese zu begehen.

Wenn die Entscheidung, diese Straftat zu begehen, allerdings allein abhängig von den zuvor beschriebenen Faktoren und den Prozessen im unterbewussten Gehirn ist, sollten Menschen eigentlich nicht für diese verantwortlich gemacht werden können. Verbrecher würde man dann nicht mehr als Strafe, sondern als Schutz für die restliche Gesellschaft inhaftieren. Gerade in der Erziehung gäbe es dann Probleme mit Bestrafungen aufgrund von Entscheidungen und Handlungen, für die ein Kind normalerweise schuldig befunden werden könnte. Demnach ist es eine gute Idee, die Illusion des freien Willens aufrecht zu erhalten.

\section{Zufall}
Um auf die Gesetze der Physik zurückzugreifen:

Laut der Physik lässt sich theoretisch alles mit Bestimmtheit vorhersagen, was sich nicht auf der vom Zufall bestimmten Quantenebene befindet. Somit ist auch alles, was nicht das Ergebnis von der Reaktion kleinster Teilchen ist, deterministisch. Darüber, ob unser Gehirn und dadurch auch unsere Entscheidungsbildung von diesen mikroskopischen Effekten betroffen ist und welche Auswirkungen diese überhaupt auf die makroskopische Welt haben, sind die Physiker noch nicht zu einem eindeutigen Konsens gekommen\footcite[S. 11]{rott}. Demnach besteht die Möglichkeit, dass nicht nur unsere Entscheidungen, sondern auch die komplette makroskopische Welt deterministisch ist und somit ein freier Wille und eine Entscheidungsfreiheit grundsätzlich nicht existieren kann.

Wenn die mikroskopischen Quanteneffekte allerdings Auswirkungen auf die makroskopische Welt und dadurch möglicherweise auch auf meine Entscheidungen hätten, wäre die einzige indeterminierte Komponente meiner Entscheidungsbildung der Zufall.

\section{Ergebnis}
Viele Versuche, meine Frage zu beantworten, basieren auf den Unklarheiten der Wissenschaft. Sowohl die physikalischen Aktivitäten in der Quantenebene, als auch die komplexen Prozesse unseres Gehirns und Bewusstseins sind häufig nicht ausreichend erforscht, beziehungsweise nicht vollständig greifbar.

Grundsätzlich gilt allerdings, dass meine Entscheidungen nicht vollkommen frei sind, sondern maßgeblich von verschiedenen Faktoren abhängen und beeinflusst werden, wodurch diese weitestgehend determiniert sind. Ein vollkommener freier Wille, und folglich auch die Entscheidungsfreiheit, ist demnach eine Illusion, die uns das alltägliche Zusammenleben erleichtert.

\newpage

\nocite{*}
\printbibliography[heading=bibintoc]

\end{document}
